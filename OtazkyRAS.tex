\documentclass{article}
\usepackage[utf8]{inputenc}

% Basic packages
	\usepackage{amssymb}
	\usepackage{amsmath}
	\usepackage{graphicx}
	\usepackage[czech]{babel}
	\usepackage{natbib}

% Relevant packages
	\usepackage{bm}
	\usepackage{bbm}
	\usepackage{geometry}
	\usepackage{enumitem}
	\usepackage{listings}
	\usepackage{multicol,float}

% Document settings
	\geometry{a4paper,margin=15mm}

	\setlist[itemize]{nolistsep,noitemsep}
		
	\setlength{\parindent}{0pt}
	\setlength{\parskip}{1.5em}
	\renewcommand{\baselinestretch}{1.2}

	\setlength{\abovedisplayskip}{1.2em}
	\setlength{\belowdisplayskip}{1.2em}
	\renewcommand{\arraystretch}{1.2}

\begin{document}
	\section{Popište možné tvary dynamických systémů pro modely: spojité-diskrétní, lineární-nelineární.}

	\subsection*{Lineární spojitý systém v časové oblasti}
	\begin{align*}
		\bm{\dot{x}} &= \bm{A}\bm{x} + \bm{B}\bm{u} \\
		\bm{y} &= \bm{C}\bm{x} + \bm{D}\bm{u}
	\end{align*}
	\subsection*{Lineární spojitý systém ve frekvenční oblasti}
	\begin{align*}
		s\bm{x} &= \bm{A}\bm{x} + \bm{B}\bm{u} \\
		\bm{y} &= \bm{C}\bm{x} + \bm{D}\bm{u}
	\end{align*}
	Po úpravě $\bm{x}(s\bm{I} - \bm{A}) = \bm{B}\bm{u}$ můžeme dosadit
	\begin{equation}
		\bm{y} = \bm{C}(s\bm{I}-\bm{A})^{-1}\bm{B}\bm{u}+\bm{D}\bm{u}
	\end{equation}
	\subsubsection*{Dynamická poddajnost}
	\begin{equation}
	\bm{G}(\bm{x}) = \frac{\bm{y}}{\bm{u}} = \bm{C}(s\bm{I}-\bm{A})^{-1}\bm{B}+\bm{D}
	\end{equation}
	\subsection*{Lineární diskrétní systém}
	\begin{align*}
		\bm{x}_{t+\Delta t} &= \bm{M}\bm{x}_t + \bm{N}\bm{u}_t \\
		\bm{y}_t &= \bm{O}\bm{x}_t + \bm{P}\bm{u}_t
	\end{align*}
	diskrétní tvar lze získat z tvaru spojitého modelu v časové oblasti
	\begin{align*}
		\bm{M} = e^{\bm{A}\Delta t} \doteq \bm{I} + \bm{A}\Delta t
		\,,\;
		\bm{N} = \bm{B}
		\,,\;
		\bm{O} = \bm{C}
		\,,\;
		\bm{P} = \bm{D}
	\end{align*}
	\subsection*{Nelineární systém}
	\begin{align}
		\bm{\dot{x}} = \bm{f}(\bm{x}) + \bm{g}(\bm{x})\bm{u} \\
		\bm{y} = \bm{c}(\bm{x}) + \bm{d}(\bm{x}) \bm{u}
	\end{align}
	
	\section{Modelování poddajných struktur}
	Většinou vycházíme z mkp modelů s $10^4 \div 10^6$ tvarů
	\begin{equation*}
		\bm{M}\bm{\ddot{x}} + \bm{C}\bm{\dot{x}} + \bm{K}\bm{x} = \bm{F}
	\end{equation*}

	Ty redukujem pomocí modální trasformace, kde hledáme řešení problému vlastních čísel
	\begin{equation*}
		\bm{K}\bm{V} = \bm{\Omega}^2 \bm{M} \bm{V}
	\end{equation*}
	kde $\bm{\phi}_i$ jsou vlastní vektory a $\omega_i$ vlastní frekvence tvořící matice $\bm{V}$ a $\bm{\Omega}$  
	\begin{equation*}
		\bm{V} = \begin{bmatrix} \bm{\phi}_i & \dots & \bm{\phi}_N \end{bmatrix}
		\,,\;
		\bm{\Omega}^2 = \operatorname{diag}(\omega_i^2)
		\;,\quad 
		i \in \langle 1,N \rangle
	\end{equation*}

	Po té platí
	\begin{equation*}
		\bm{V}^T\bm{M}\bm{V} = \bm{1}
		\;,\quad 
		\bm{V}^T\bm{K}\bm{V} = \bm{\Omega}^2
	\end{equation*}

	Soustavu pohybovných rovnic systému s proporčním tlumením

	lze zavedením modální souřadnic $\bm{q} \,,\; \bm{x} = \bm{V}\bm{q}$ a vynásobením transponovanou maticí modální transformace $\bm{V}^T$ zleva, převést do tvaru
	\begin{equation*}
		\bm{\ddot{q}} + \bm{\beta}\bm{\dot{q}} + \bm{\Omega}^2 \bm{q} = \bm{V}^T \bm{F}
		\;,\quad 
		\bm{\beta} = \operatorname{diag}(2\,b_{r_i}\omega_i) \;,\quad i \in \langle 1,N \rangle
	\end{equation*}
	kde $ b_{r_i}$ jsou poměrné útlumy.

	Soustava se pak rozpadá na rovnice ve tvaru
	\begin{equation*}
		\ddot{q}_i + 2\,\omega_i\xi_i \dot{q} + \omega_i^2 q = f_i
		\;,\quad 
		f_i = \bm{\phi}_i \cdot \bm{F}
		\;,\quad 
		i \in \langle 1,N \rangle
	\end{equation*}
	kde bere rovnice pro $10^1 \div 10^2$ nejnižších vlastních frekvencí.

	\section{Redukce modelů poddajných struktur pro syntézu řízení. Zohlednění vypuštěných stavů soustavy pomocí reziduí.}

	\section{Balancovaný tvar stavového popisu soustavy, smysl tohoto tvaru při získání návrhového modelu.}

	\section{Typy aktivních a poloaktivních aktuátorů používaných v mechatronických systémech, hlavní vlastnosti, důsledky pro použití.}

	\section{Snižování vibrací (low-authority) versus řízení pohybu/polohy (high-authority). Princip, příklady}

	\section{Aktivní absorbce kmitů soustav, příklad návrhu řízení}

	\section{Mechatronická tuhost. Princip, model, příklad návrhu řízení.}

	\section{Aktivní tlumení vibrací pomocí kolokovaných aktuátorů a senzorů. Příklad s použitím planárních piezo-aktuátorů.}

	\section{Aktivní vibroizolace soustav, integrální silová zpětná vazba, příklad návrhu řízení.}

	Nahradíme piezostackem s tuhostí $k$, nastavitelnou volnou délkou $l_0(u) = l_{00} + qu$ a pozitivní, integrální, silovou zpětnou vazbou $u = p \int F\,dt$, kde $p$ je volitelný parametr.

	Pohybová rovnice systému bude nabývat tvar
	\begin{equation}
		m\ddot{y} = - \underbrace{k ( y-z_0(t)-qu)}_{F}
	\end{equation}
	Dosazením z pohybové rovnice můžeme určit tvar akčního zásahu
	\begin{align}
		u = p \int F\,dt = p \int -m\ddot{y} = -pm\dot{y} + C
	\end{align}

	Dosazením tvaru akčního zásahu zpět do pohybové rovnice získáme pohybovou rovnici tlumeného systému
	\begin{equation}
		m\ddot{y} = -k (y-z_0(t)) - \underbrace{kmpq}_{b_{\text{sky}}}\dot{y} - kqC
	\end{equation}
	kde hodnotu $b_{\text{sky}}$ můžeme ladit volbou parametru $p$

	\section{Říditelnost a pozorovatelnost, Gramián říditelnosti a pozorovatelnosti}

	Uvažujme lin. systém ve tvaru
	\begin{align*}
		\bm{\dot{x}} &= \bm{A}\bm{x} + \bm{B}\bm{u} \\
		\bm{y} &= \bm{C}\bm{x} + \bm{D}\bm{u}
	\end{align*}

	\subsection*{Říditelnost}
	Systém je říditelný pokud jej lze z libovolného stavu $\bm{x} \in \mathbb{R}^n$ dostat do stavu nulového $\bm{x} = \bm{0}$ působením vstupů $\bm{u}$.

	\subsubsection*{Matice říditelnosti}
	\begin{equation}
		\bm{\mathcal{C}}
		=
		\begin{bmatrix}
			\bm{B} & \bm{A} \bm{B} & \bm{A}^{2} \bm{B} & \dots & \bm{A}^{n-1} \bm{B}
		\end{bmatrix}
	\end{equation}
	Pokud $\bm{\mathcal{C}}$ je plné hodnosti, systém je říditelný.

	\subsubsection*{Gramián říditelnosti}
	\begin{equation}
		\bm{W}_{c}
		=
		\int_{0}^{\infty} e^{\bm{A} \tau} \bm{B} \bm{B}^T e^{\bm{A}^T \tau} d \tau
	\end{equation}
	Vlastní vektory $\bm{W}_c$ patřící k největším vlastním číslům jsou nejlépe říditelné směry ve stavovém prostoru. Podmíněností $\bm{W}_c$ můžeme hodnotit celkovou říditelnost systému (ve všech směrech stavového prostoru).

	\subsection*{Pozorovatelnost}
	Systém je pozorovatelný pokud na konečném časovém intervalu lze ze změřeného průběhu vstupů $\bm{u}$ a výstupů $\bm{y}$ určit stav systému na počátku invervalu $\bm{x}_0 \in \mathbb{R}^n$.
	
	\subsubsection*{Matice pozorovatelnosti}
	\begin{equation}
		\bm{\mathcal{O}}
		=
		\begin{bmatrix}
			\bm{C} \\
			\bm{C} \bm{A} \\
			\bm{C} \bm{A}^{2} \\
			\vdots \\
			\bm{C} \bm{A}^{n-1}
		\end{bmatrix}
	\end{equation}
	Pokud $\bm{\mathcal{O}}$ je plné hodnosti, systém je pozorovatelný.

	\subsubsection*{Gramián pozorovatelnosti}
	\begin{equation}
		\bm{W}_{o}
		=
		\int_{0}^{\infty} e^{\bm{A} \tau} \bm{C} \bm{C}^T e^{\bm{A}^T \tau} d \tau
	\end{equation}
	Vlastní vektory $\bm{W}_o$ patřící k největším vlastním číslům jsou nejlépe pozorovatelné směry ve stavovém prostoru. Podmíněností $\bm{W}_o$ můžeme hodnotit celkovou pozorovatelnost systému (ve všech směrech stavového prostoru).

	\section{Computed Torques}
	\begin{equation}
		\bm{\ddot{e}}
		=
		\bm{\ddot{q}}_{d}-\bm{\ddot{q}}
		=
		\bm{\ddot{q}}_{d}-\bm{M}^{-1} \bm{\tau}+\bm{M}^{-1} \bm{N}
	\end{equation}

	\begin{equation}
		\frac{d}{d t}
		\begin{bmatrix}
			\bm{e} \\
			\bm{\dot{e}}
		\end{bmatrix}
		=
		\begin{bmatrix}
			\bm{0} & \bm{I} \\
			\bm{0} & \bm{0}
		\end{bmatrix}
		\bm{x}
		+	
		\begin{bmatrix}
			\bm{0} \\
			\bm{I}
		\end{bmatrix}
		\bm{u}
	\end{equation}

	\section{Použití optimalizačních metod pro syntézu řízení. Aktivní a poloaktivní aktuátory, lineární a nelineární soustavy (ilustrace na příkladu nelineární soustavy).}


\end{document}